\documentclass[12pt]{article}
\usepackage[margin=2.5cm]{geometry}
\usepackage{enumerate, fancyhdr, graphicx, amsmath, lipsum, wrapfig, nameref, url, hyperref,
            setspace, color, csquotes, textcomp, subcaption, float }

\title{Mars Landing Site Traversability}
\author{Paul Chesnais (pmc85)}
\date{\today}

\pagestyle{fancy}
\fancyhead{}
\lhead{Paul Chesnais (pmc85)}
\chead{Mars Landing Site Traversability}
\rhead{\today}
\fancyfoot{}
\rfoot{\thepage}
\lfoot{\includegraphics[height=20pt]{Logo}}
\renewcommand{\headrulewidth}{0.5pt}
\renewcommand{\footrulewidth}{0.5pt}

\usepackage[binary-units=true]{siunitx}
\sisetup{load-configurations = abbreviations}

\newcommand{\source}{\href{https://github.com/PapaCharlie/Rover-Climb-Angles/tree/master/figures/maps/}{\ttfamily\color{blue} \underline{source}}}
\newcommand{\p}[1]{\times 10^{#1}}
\newcommand{\supcite}[1]{\textsuperscript{\cite{#1}}}
\newcommand{\MATLAB}{MATLAB\textsuperscript{\textregistered}}

\begin{document}
\maketitle
\thispagestyle{empty}

\section{Abstract}
\label{sec:abstract}
Exploring a planet from the ground isn't easy. Especially since as of today, rovers are the only things capable of doing it. Unfortunately, designing rovers is incredibly difficult. Given the topology for a candidate landing site, how can you judge its ``quality'' or difficulty? How should you construct the rover to optimize it for traversing the site? This project aims to illustrate how agile Mars rovers should be by computing traversability maps of candidate landing sites.

\section{Datasets}
\label{sec:datasets}
\par For the purposes of this project, the data must be accurate on a scale that is reasonable relative to the Rover's size. Curiosity was 3 meters long\supcite{bib:curiosity}, and the Mars 2020 rover is also expected to be 3 meters long\supcite{bib:rover2020}. This dictates the order of the resolution of the datasets required for this study to be meaningful. Suppose a topographical map with a horizontal resolution on the scale of 20 meters or even 10 meters. Unfortunately, insurmountable rocks and obstacles come in all shapes and sizes, many of which smaller than 20 meters. Such a map would not necessarily reflect the presence of these obstacles and pose an unforeseen problem to rover. This effectively leaves two choices in topographical maps: HSRC Digital Topographical Maps (DTMs) from the Mars Express mission, or HiRISE DTMs from the Mars Reconnaissance Orbiter mission. HSRC DTMs are on the scale of 2 meters per pixel\supcite{bib:hsrc} and HiRISE DTMs are on the accuracy of 1 meter per pixel\supcite{bib:abouthirise}. For the sake of this study's rigor, the more recent and more accurate HiRISE DTMs were used. Given that the horizontal resolution of those DTMs is three times smaller than the length of the rover and that their vertical accuracy is ``in the tens of centimeters''\supcite{bib:abouthirise}, any unforeseen obstacles should be small, flat, and, in theory, relatively easy to avoid.
\par Figure~\ref{fig:dtms} is a table showing which datasets were used, their respective identifiers, resolutions and area of coverage. It is noted that not all candidate landing sites were covered and presented in this study, but the algorithm is robust and is designed to be used with all HiRISE DTMs. The datasets were selected for their interesting features and potentially treacherous terrain. Please visit the code's \source{} for more maps.
\par The number of pixels is quoted to highlight the complexity of the datasets. Additionally, each pixel is stored as 64-bit floating point number in memory. In other words, each dataset occupies more than \SI{1}{\giga\byte} of RAM, making them tricky to work with in most computer languages.
\par The value of each pixel represents its vertical distance from a theoretical, well defined sphere surrounding Mars. Not every DTM uses a sphere of the same radius, but each DTM is consistent throughout. Therefore subtracting the value of one pixel from its neighbor gives the difference in height between the two, without any further transformations.
% \par The average area covered is $\SI{103.15}{\kilo\meter\squared}$, which may seem small, but given that Curiosity's ground speed is about $\SI{0.14}{\kilo\meter\per\hour}$, it would take close to 500 days nonstop to visit every single point.
\begin{figure*}
  \center
  \begin{tabular}[b]{c|c|c|c|c}
    Name & Dataset & Pixels & Resolution & Area covered \\ \hline
    Southwest Arabia Terra &  ESP\_011844\_1855\supcite{bib:ESP_011844_1855} & 1.74e+08 & 1.011655 & $\SI{107.39}{\kilo\meter\squared}$\\
    Gale Crater &             ESP\_023957\_1755\supcite{bib:ESP_023957_1755} & 1.14e+08 & 1.011763 & $\SI{79.47}{\kilo\meter\squared}$\\
    Holden Crater &           ESP\_019612\_1535\supcite{bib:ESP_019612_1535} & 1.90e+08 & 1.007219 & $\SI{122.50}{\kilo\meter\squared}$\\
    Eberswalde Crater &       ESP\_019757\_1560\supcite{bib:ESP_019757_1560} & 1.64e+08 & 1.008208 & $\SI{110.22}{\kilo\meter\squared}$\\
    Mawrth Vallis &           ESP\_015985\_2040\supcite{bib:ESP_015985_2040} & 1.33e+08 & 1.008208 & $\SI{96.15}{\kilo\meter\squared}$\\
  \end{tabular}
  \caption{DTMs used and their respective properties}
  \label{fig:dtms}
\end{figure*}

\section{Method}
\label{sec:method}
\subsection{Dijkstra's Algorithm}
\label{sub:the_algorithm}
\par The algorithm used was inspired by the single-source version of Dijkstra's algorithm. Dijkstra's algorithm is an algorithm used to find the shortest path from the starting node to a given node. The single-source version is a generalization that finds the shortest path from a given starting node to all the nodes in a graph. This modified version of the algorithm finds the flattest path between two points, i.e. the path that requires the flattest climbing angle. The ``required'' angle, in this case, would be the angle of the steepest slope along the path between the two points.
\par Suppose a graph $G$ representing a topographical map, with $V$ the set of all nodes in the graph, and $E$ the set of all edges. Each node in the graph represents a physical location and its associated altitude. For notation, let the height of a given node $u$ be returned by $height(u)$, the function $d(u.v)$ be the function that returns the physical distance between $u$ and $v$, and the function $angle(u,v)$ be the function that returns the angle of the slope between points $u$ and $v$. In this case, for two nodes $u$ and $v$:
\begin{align*}
  angle(u,v) = \tan^{-1}\left(\frac{height(u) - height(v)}{d(u,v)}\right)
\end{align*}
Finally, for any two connected nodes $u$ and $v$, the value of the edge $(u,v)$ that connects them is $angle(u,v)$.
\pagebreak
\par A data structure that stores pairs of nodes and angles according to a certain ordering is also required. In this case, the functions associated with what will be called a Frontier are defined as follows, where $fr$ is an arbitrary Frontier:
\begin{itemize}
  \item For a node $v$ and an angle $\alpha$, let $push(fr, (v,\alpha))$ be the function that adds the pair $(v, \alpha)$ to the Frontier $fr$. If there exists a pair $(v,\beta)$ in $fr$ such that $|\alpha| \leq |\beta|$, remove the pair $(v,\beta)$ and add the pair $(v,\alpha)$.
  \item Let $pop(fr)$ return the pair $(v, \alpha)$ such that for pairs $(u,\beta)$ in $fr$: $|\alpha| \leq |\beta|$.
  \item Let $\|fr\|$ be the number of pairs contained in $fr$.
\end{itemize}
\par The modified Dijkstra's algorithm can be expressed using the above definitions and an arbitrary starting node $s$:
\begin{enumerate}[Step 1]
  \item Define an array $required$ such that $required[v]$ returns the flattest required angle for the paths that connect $s$ to $v$. Because no nodes have been visited yet, set $required[v] = +\infty$ for all nodes $v$ in the set $V$. Define an empty Frontier $fr$.
  \item Visit $s$, i.e. execute $push(fr, (s, 0))$ and set $required[s] := 0$.
  \item Let $(v,\alpha) = pop(fr)$. For all each neighbor $n$ of $v$, do the following:
  \begin{displayquote}
    If $|required[v]| < |angle(v,n)|$, set $\alpha := angle(v,n)$, otherwise set $\alpha := required[v]$.\\
    If $ |\alpha| < |required[n]|$, set $required[n] := \alpha$ and $push(fr, (n, \alpha))$. Otherwise, do nothing.
  \end{displayquote}
  Repeat Step 3 until $|fr| = 0$
\end{enumerate}
\par For this algorithm, the Frontier represents the the set of nodes that can be reached from the set of explored nodes. By \emph{popping} the flattest pair from the Frontier every time, the guarantee that the associated required angle for each node on the frontier is the flattest possible holds. Suppose two neighboring nodes $u$ and $v$, such that $required[v] = 0$, $|angle(v,u) > 0|$, and $u$ has not yet been visited. The value at $required[u]$ must be set to $angle(v,u)$, otherwise $u$ will inherit a flatter angle than the true required angle. Therefore Step 3 maintains the correctness of the traversal by setting the value in $required$ to be whichever angle is steeper between $required[v]$ and $angle(v,n)$. With the guarantees maintained by Step 3 and the Frontier data structure, every time a required angle is associated to a given node, it is the correct required angle. Additionally, the algorithm is guaranteed to halt because no node is revisited once it has been popped from the Frontier, and each node is added iteratively. Since there are a finite number of nodes, the algorithm must complete. The final $required$ therefore contains the correct values for every single node.

\subsection{Using the data}
\label{sub:using_the_data}
\par First, the data needs to be represented as the Graph explained above. In this case, each pixel is a node $p$ and $height(p)$ is simply the value in the DTM at the position $p$. Next, each pixel is connected to its 4 closest neighbors, i.e. the pixels that are directly in contact with it. To avoid interpolation and potential errors, pixels are not connected to diagonal neighbors. As such, $d(u,v)$ be set to always be equal to the horizontal resolution in the dataset for any two neighboring pixels $u$ and $v$. The starting position $s$ will be naively chosen to be the pixel at the center of the DTM. For this, there is no need to actually implement a Graph data structure, since a two dimensional array of values is already implicitly a rectangular graph.
\par Next, the Frontier data structure needs to be implemented to the same specifications as defined above. In this case, a MinHeap is ideal because it fits the exact criteria. A MinHeap is a balanced data structure, where $push$ and $pop$ operations require that the MinHeap be rebalanced. If there are $n$ elements in the MinHeap, then rebalancing the MinHeap will take at worst $\log_2(n)$ operations. This means that finding the flattest pair in the MinHeap can be done very quickly with little overhead. Since the DTMs usually have more than 100 million pixels and the frontier of visited pixels can grow arbitrarily large, fast $push$ and $pop$ operations become crucial. Additionally, the ordering between pairs can be defined at will, meaning that a MinHeap can be implemented to order the pairs exactly as required.
\par Finally, all that is left is to create the $required$ array. This is easily done by creating an array the same size as the DTM and indexing into it exactly like the DTM is indexed into. Unfortunately, these operations become computationally expensive quite quickly. If the DTM takes 1 GB in memory, so will the $required$ array. Additionally, as said before, the Frontier can grow arbitrarily large. As such, in the absolute worst case scenario, computing the traversability map can require up to 3 GB of memory at runtime. This causes a lot of issues because not all of the data can be loaded into the processor's more restricted memory at a time, meaning that a lot of time is spent waiting for data from RAM. Fortunately, Modern RAM responds quite quickly and data usually will only take a handful of CPU cycles to arrive.

\subsection{Optimizations}
\label{sub:optimizations}
\par There was an initial, fruitless attempt at implementing this in \MATLAB{}. To illustrate the speed of the process, the program gave periodic progress updates in the form of messages saying that a certain integer percent of the pixels had been visited. Unfortunately, these progress updates came at the rate of about once every hour, leading one to believe that iterating over all of the datasets could, and most likely would, take weeks. Clearly, a radical change need to be made.
\par As such, \MATLAB{} was dropped, and Golang became the language of choice. Because Golang is compiled, the same program was expected to run more efficiently. The current implementation of the algorithm takes less than 2 minutes to compute the entire traversability map. An immense, unexpected, yet welcome change in runtime which meant that computation could move forward at a more reasonable pace.
% Other, smaller, optimizations were made. For example, computing trigonometric functions is computationally expensive because it requires approximating an infinite sum. As such, instead of computing the arc tangent of the difference in height between two neighbors over the resolution, only the difference in height was recorded.

\subsection{Data Pipeline}
\label{sub:data_pipeline}
\par HiRISE DTMs come in the form of IMG files. This meant that they needed to be converted into a format that was readable by Golang. Golang already had a library that can read FITS files. As such, the USGS ISIS software package was used to convert the IMG files into FITS files. Then it is a simple matter of extracting the resolution and image size from the label included in the IMG file to get all the information required to compute the map.
\par Once the algorithm has run its course, it saves the map to disk as a simple binary dump of the array, which \MATLAB{} is capable of reading. The map is loaded into memory and stretched between -20\textdegree and 20\textdegree to show detail, under the assumption that rovers cannot, or prefer not to, climb slopes past those angles. Finally, what will be called a traversability histogram is plotted. A traversability histogram is a histogram that shows required angle versus number of pixels. To find the area a given rover can cover, integrate the histogram between the bounds $\alpha$ and $\beta$ where $\alpha$ and $\beta$ are respectively the steepest negative and positive angles the rover can climb. The shaded area represents the range of required angles to access 95\% of the landing site.

\section{Results}
\label{sec:results}
% \par We can finally begin presenting the results of our hard work.
\subsection{Southwest Arabia Terra\supcite{bib:ESP_011844_1855}}
\label{sub:southwest_arabia_terra}
\begin{figure}[h!]
  \centering
  \begin{subfigure}[t]{0.27\textwidth}
    \centering
    \includegraphics[height=0.4\paperheight]{figures/maps/ESP_011844_1855/DTEEC_011844_1855_002812_1855_A01.jpg}
    \caption{Color altimetry map}
    \label{fig:southwest_dtm}
  \end{subfigure}
  ~
  \begin{subfigure}[t]{0.27\textwidth}
    \centering
    \includegraphics[height=0.4\paperheight]{figures/maps/ESP_011844_1855/DTEEC_011844_1855_002812_1855_A01-traversability_map.pdf}
    \caption{Traversability map}
    \label{fig:southwest_traversability}
  \end{subfigure}
  \caption{Southwest Arabia Terra Landing Site Traversability Map}
  \label{fig:southwest}
\end{figure}
\par Figure~\ref{fig:southwest_dtm} and Figure~\ref{fig:southwest_traversability} respectively show the colored DTM, and the output traversability map for the Southwest Arabia Terra landing site. This map shows some rather interesting features. As we can see, the dark blue and the dark red in lower third of the site seem to indicate that the area is completely inaccessible. There is a very clear dividing line between this third and the rest of the map. This line is reflected in the DTM, where there is a cliff face along it. It does appear that the algorithm highlights these features and handled the cliff face rather well. As for the upper half of the map, there also seems to be a less severe divide between the low basin in the middle, and the higher terrain. The rover needs to be able to climb at least 5\textdegree to access this area. Looking at the traversability histogram in Figure~\ref{fig:southwest_hist}, we see that there are still a lot of pixels beyond the $\pm$20\textdegree bound, and a large segment of the pixels are located between +10\textdegree and +20\textdegree, i.e. the orange band just north the inaccessible area.
\begin{figure}[h!]
  \centering
  \includegraphics[width=0.6\textwidth]{figures/maps/ESP_011844_1855/DTEEC_011844_1855_002812_1855_A01-hist.pdf}
  \caption{Traversability Histogram for the Southwest Arabia Terra Landing Site}
  \label{fig:southwest_hist}
\end{figure}

\subsection{Mawrth Vallis\supcite{bib:ESP_015985_2040}}
\label{sub:mawrth_vallis}
\begin{figure}[h!]
  \centering
  \begin{subfigure}[t]{0.35\textwidth}
    \centering
    \includegraphics[height=0.4\paperheight]{figures/maps/ESP_015985_2040/DTEEC_015985_2040_016262_2040_U01.jpg}
    \caption{Color altimetry map}
    \label{fig:mawrth_dtm}
  \end{subfigure}
  ~
  \begin{subfigure}[t]{0.35\textwidth}
    \centering
    \includegraphics[height=0.4\paperheight]{figures/maps/ESP_015985_2040/DTEEC_015985_2040_016262_2040_U01-traversability_map.pdf}
    \caption{Traversability map}
    \label{fig:mawrth_traversability}
  \end{subfigure}
  \caption{Mawrth Vallis Landing Site Traversability Map}
  \label{fig:mawrth}
\end{figure}
\par This section of the Mawrth Vallis landing site was chosen because of its odd topography. The DTM in Figure~\ref{fig:mawrth_dtm} suggests that there are three distinct plateaus, and no guarantee that there exists a climbable path connecting them. In other words, if the rover were to land in the lowest plateau, it would never be able to gain access to the remaining ones, and vice versa. Fortunately, the traversability map in Figure~\ref{fig:mawrth_traversability} informs us that the entire section is in fact quite flat. Almost all of the traversability map is shaded in yellow or light blue, implying that those areas can be reached with a required angle of $\pm$5\textdegree. Looking at the histogram in Figure~\ref{fig:mawrth_hist}, 95\% of the landing site can be covered with a little less than $\pm$10\textdegree.
\begin{figure}[h!]
  \centering
  \includegraphics[width=0.6\textwidth]{figures/maps/ESP_015985_2040/DTEEC_015985_2040_016262_2040_U01-hist.pdf}
  \caption{Traversability Histogram for the Mawrth Vallis Landing Site}
  \label{fig:mawrth_hist}
\end{figure}

\subsection{Eberswalde, Gale and Holden Craters}
Please see figures \ref{fig:eberswalde}, \ref{fig:eberswalde_hist}, \ref{fig:gale}, \ref{fig:gale_hist}, \ref{fig:holden} and \ref{fig:holden_hist} in the Appendix for the DTMs, traversability maps and traversability histograms of the candidate landing sites in Eberswalde, Gale and Holden craters. Additionally, the code's \source{} contains studies of more candidate sites.

\section{Weaknesses and Future Work}
\label{sec:weaknesses_and_future_work}
\par There are two weaknesses in the current version of the algorithm.
\par As mentioned in the algorithm's definition, the starting position is always picked to be the middle of the DTM. Essentially, this means that the required angles shown in the map are actually the angles required to access those points had we started from the center. Fortunately, a lot of information can still be inferred from these maps, even with this slight subtlety. One of the ways to fix this problem would be to use as a starting point the actual target area for the site. A better method would be to pick multiple starting positions. In this case, the frontier of reachable points would start disconnected, then merge with its cousins as the algorithm progresses. This would give a better understanding of the terrain.
\par Another weakness in this algorithm is that it is a little naive. The difficulty in getting from point A to point B does not only lie in the angle of the slope between A and B. It also lies within the actual material the rover is going over. A slope of sand is harder to go over than a slope of gravel or rock. As such, one of the ways to improve this algorithm would be to keep track of a notion of ``difficulty'', as opposed to only the angle. Suppose we had a thermal inertia map of the terrain at a resolution close to the resolution of the DTM. Given a mapping between thermal inertia and terrain traversability, the difficulty between two points can be expressed as the combination of the angle and the traversability. With such data, we would have a much better approximation of the true difficulty of the site. Current thermal inertia maps based on data from the Mars Reconnaissance Orbiter have an accuracy of \SI{100}{mpp}, meaning that there would be approximately 100 data points per landing site. Though that seems like few, using those maps would still provide some valuable information.
\par As long as the data is provided, both of these modifications are implementable without too much overhead.

\section{Conclusion}
\label{sec:conclusion}
\par The goal of this project was to establish a pipeline takes DTMs at arbitrary precisions, and outputs the respective traversability maps in a reasonable. Given the more than successful results with the candidate landing sites, we can safely conclude that this goal has been achieved. The final product is a stable piece of software that outputs valuable information, though there is still room for improvement.
\par As for the landing sites presented in this paper, it would seem that Mawrth Vallis is a great candidate. It is flat and traversable, and given that it is one of the oldest valleys on Mars, is certainly a region of interest. On the other hand, landing sites such as the one near Southwest Arabia Terra appear to be much harder to traverse. Though it may contain valuable information, it is not necessarily a good choice if movement is hindered every step of the way. The traversability maps were created for the sole purpose of aiding the in the choice of landing site and rover design, and it appears that this pipeline would indeed be a useful tool in making those decision.

\clearpage
\begin{thebibliography}{30}
\bibitem{bib:rover2020}
  \url{http://mars.nasa.gov/mars2020/mission/rover/}
\bibitem{bib:curiosity}
  \url{http://mars.nasa.gov/msl/mission/rover/}
\bibitem{bib:hirise}
  \url{http://mars.nasa.gov/mro/mission/instruments/hirise/}
\bibitem{bib:abouthirise}
  \url{http://www.uahirise.org/dtm/about.php}
\bibitem{bib:nasa-review}
  \url{http://www.marsjournal.org/contents/2008/0002/files/som_mars_2008_0002.pdf}
\bibitem{bib:angle}
  \url{http://www.jpl.nasa.gov/news/news.php?feature=4596}
\bibitem{bib:hsrc}
  \url{http://pds-geosciences.wustl.edu/missions/mars_express/hrsc.htm}
\bibitem{bib:groundspeed}
  \url{http://www.space.com/22226-mars-rover-curiosity-time-lapse-video.html}
\bibitem{bib:ESP_011844_1855}
  \url{www.uahirise.org/dtm/dtm.php?ID=ESP_011844_1855}
\bibitem{bib:ESP_023957_1755}
  \url{www.uahirise.org/dtm/dtm.php?ID=ESP_023957_1755}
\bibitem{bib:ESP_019612_1535}
  \url{www.uahirise.org/dtm/dtm.php?ID=ESP_019612_1535}
\bibitem{bib:ESP_019757_1560}
  \url{www.uahirise.org/dtm/dtm.php?ID=ESP_019757_1560}
\bibitem{bib:ESP_015985_2040}
  \url{www.uahirise.org/dtm/dtm.php?ID=ESP_015985_2040}
\bibitem{bib:thermal_interta}
  \url{http://astrogeology.usgs.gov/maps/mars-themis-derived-global-thermal-inertia-mosaic}
\end{thebibliography}

\section{Appendix}
\label{sec:appendix}
\begin{figure}[h!]
  \centering
  \begin{subfigure}[t]{0.35\textwidth}
    \centering
    \includegraphics[height=0.4\paperheight]{figures/maps/ESP_019757_1560/DTEEC_019757_1560_020034_1560_U01.jpg}
    \caption{Color altimetry map}
    \label{fig:eberswalde_dtm}
  \end{subfigure}
  ~
  \begin{subfigure}[t]{0.35\textwidth}
    \centering
    \includegraphics[height=0.4\paperheight]{figures/maps/ESP_019757_1560/DTEEC_019757_1560_020034_1560_U01-traversability_map.pdf}
    \caption{Traversability map}
    \label{fig:eberswalde_traversability}
  \end{subfigure}
  \caption{Mawrth Vallis Landing Site Traversability Map}
  \label{fig:eberswalde}
\end{figure}
\begin{figure}[h!]
  \centering
  \includegraphics[width=0.6\textwidth]{figures/maps/ESP_019757_1560/DTEEC_019757_1560_020034_1560_U01-hist.pdf}
  \caption{Traversability Histogram for the Mawrth Vallis Landing Site}
  \label{fig:eberswalde_hist}
\end{figure}

\begin{figure}[h!]
  \centering
  \begin{subfigure}[t]{0.35\textwidth}
    \centering
    \includegraphics[height=0.4\paperheight]{figures/maps/ESP_023957_1755/DTEEC_023957_1755_024023_1755_U01.jpg}
    \caption{Color altimetry map}
    \label{fig:gale_dtm}
  \end{subfigure}
  ~
  \begin{subfigure}[t]{0.35\textwidth}
    \centering
    \includegraphics[height=0.4\paperheight]{figures/maps/ESP_023957_1755/DTEEC_023957_1755_024023_1755_U01-traversability_map.pdf}
    \caption{Traversability map}
    \label{fig:gale_traversability}
  \end{subfigure}
  \caption{Mawrth Vallis Landing Site Traversability Map}
  \label{fig:gale}
\end{figure}
\begin{figure}[h!]
  \centering
  \includegraphics[width=0.6\textwidth]{figures/maps/ESP_023957_1755/DTEEC_023957_1755_024023_1755_U01-hist.pdf}
  \caption{Traversability Histogram for the Mawrth Vallis Landing Site}
  \label{fig:gale_hist}
\end{figure}

\begin{figure}[h!]
  \centering
  \begin{subfigure}[t]{0.35\textwidth}
    \centering
    \includegraphics[height=0.4\paperheight]{figures/maps/ESP_019612_1535/DTEEC_019612_1535_019678_1535_U01.jpg}
    \caption{Color altimetry map}
    \label{fig:holden_dtm}
  \end{subfigure}
  ~
  \begin{subfigure}[t]{0.35\textwidth}
    \centering
    \includegraphics[height=0.4\paperheight]{figures/maps/ESP_019612_1535/DTEEC_019612_1535_019678_1535_U01-traversability_map.pdf}
    \caption{Traversability map}
    \label{fig:holden_traversability}
  \end{subfigure}
  \caption{Mawrth Vallis Landing Site Traversability Map}
  \label{fig:holden}
\end{figure}
\begin{figure}[h!]
  \centering
  \includegraphics[width=0.6\textwidth]{figures/maps/ESP_019612_1535/DTEEC_019612_1535_019678_1535_U01-hist.pdf}
  \caption{Traversability Histogram for the Mawrth Vallis Landing Site}
  \label{fig:holden_hist}
\end{figure}

\end{document}
